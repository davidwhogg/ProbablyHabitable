\documentclass[12pt,letterpaper]{article}

\begin{document}

\begin{abstract}
The total number of potentially habitable exoplanets
  (rocky planets with liquid surface water)
  that can be followed up spectroscopically within the next decade
  will be severely limited
  by the availability of space-based telescope facilities.
Both because star and exoplanet observations are noisy,
  and because calculations of planetary atmospheres and surfaces are uncertain,
  the question of whether a planet is habitable%
  ---and in particular in any way that can be spectroscopically confirmed---%
  will not be answerable with certainty for any individual system.
The community will face tough decisions about what to observe,
  given the limited resources and the tentative nature of any target list.
Here we advocate analysis of this problem in the framework of \emph{decision theory};
  this requires computation of posterior probability densities
  over exoplanet properties related to habitability,
  and some ability to quantify the utility of various kinds of observational outcomes.
For the posterior probability computation,
  we propose a hierarchical probabilistic approach.
For the utility, we are at a loss (as it were).
We build and execute the necessary posterior probability machinery for the
  habitable-zone planet XXX.
We find YYY.
\end{abstract}

\section{Introduction}

...Prospects for JWST observations of rocky planets...

...There are habitable-zone candidates now...

...Habitability status changes as understandings about host-star properties and exoplanet atmospheres evolve...

...Hard choices are in store...

\section{Decision theory}

\section{Hierarchical inference}

\section{Application to XXX}

\section{Discussion}

\end{document}
